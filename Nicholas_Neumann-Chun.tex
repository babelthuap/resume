% LaTeX resume using res.cls
\documentclass[margin]{res}
%\usepackage[a4paper,hmargin=1.0in,vmargin=1.0in]{geometry}
%\usepackage{url}
\setlength{\textwidth}{5.1in} % set width of text portion
%\setlength{\resumewidth}{6.5in} % set width of whole resume


% Draw a think horizontal line the whole width of resume:
\newcommand{\HL}{\moveleft\hoffset\vbox{\hrule width\resumewidth height 1pt}}
% Draw a thin horizontal line between sections:
\newcommand{\hl}{\moveleft\hoffset\vbox{\hrule width\resumewidth}\vspace{-0.11in}}


\begin{document}

% Center the name over the entire width of resume:
 \moveleft.5\hoffset\centerline{\huge\bf Nicholas G. Neumann-Chun}
 \vspace{0.08in}
 \moveleft.5\hoffset\centerline{\large\sl Full-Stack JavaScript Developer; Mathematician with a degree from Williams College}
% \smallskip

\vspace{-0.02in} 
\HL % Horizontal line
\vspace{-0.12in}

% addresses begin here
\address{ 349 Harvard Common \\ Fremont, CA  94539 \\ (651) 491-4928 }
\address{nicholas.babelthaup@gmail.com \\ @Babelthuap \\ https://babelthuap.github.io }

% begin the resume content
\begin{resume}

\vspace{-0.1in}
\bigskip
\HL
\vspace{-0.13in}

\section{COMPUTER}
	{\bf Skills:} JavaScript, Node, React, Angular, Express, MongoDB, jQuery, Git, Gulp, Heroku, Bootstrap, \LaTeX
	
	\vspace{-0.1in}
	
	{\bf Exposure:} Java, Python, Scala, Mathematica, Flux, GraphQL, Relay, Firebase, jspm, Webpack, Mocha, Passport, Foundation 


\hl\section{EXPERIENCE}
	{\bf Full-Stack Developer and Code Mentor}, Coding House \hfill {\sl since Jan 2016}
	\begin{itemize}  \itemsep -2pt
	\item Worked in teams creating full-stack JavaScript apps
	\item Mentored students on topics including Git and all MEAN technologies
	\item Reviewed, graded, and provided feedback on student projects
	\end{itemize}
	{\bf Math \& Physics Teaching Assistant} \hfill {\sl 2009-2013}
	\begin{itemize}  \itemsep -2pt
	\item While a student at Williams College
	\item As a TA for various classes, held weekly workshops and graded homework
	\item Tutored students one-on-one
	\end{itemize}


\hl\section{COOL \\ PROJECTS}
	{\bf Start Coding} -- http://robertsonsamuel.github.io/startcoding-frontend \hfill {\sl Feb 2016}
	\begin{itemize}  \itemsep -2pt
	\item A public, social bookmarks list for discovering and sharing coding resources
	\end{itemize}
	{\bf Green it!} -- http://paulgoblin.github.io/greenit-frontend \hfill {\sl Jan 2016}
	\begin{itemize}  \itemsep -2pt
	\item A Reddit-inspired app built with ReactJS and MongoDB
	\end{itemize}
	{\bf Friend Finder} -- http://young-favorite-users.herokuapp.com \hfill {\sl Jan 2016}
	\begin{itemize}  \itemsep -2pt
	\item A Facebook-inspired, full-stack MEAN app hacked together in less than a week
	\end{itemize}


\hl\section{PUBLICATIONS}
	Garrity, Thomas. {\it Electricity and Magnetism for Mathematicians: A Guided Path from Maxwell's Equations to Yang-Mills}. New York: Cambridge University Press, 2015.
	\begin{itemize}  \itemsep -2pt
	\item Created all diagrams, including cover illustration, with Adobe Illustrator
	\item Proofread, indexed, and worked all exercises
	\end{itemize}
	
	Krishna Dasaratha, Laure Flapan, Thomas Garrity, Chansoo Lee, Cornelia Mihaila, Nicholas Neumann-Chun, Sarah Peluse, Matthew Stoffregen. ``A Generalized Family of Multidimensional Continued Fractions: TRIP Maps." {\it International Journal of Number Theory} 10.8 (2014): 2151-2186. http://arxiv.org/abs/1206.7077
	\begin{itemize}  \itemsep -2pt
	\item One result of the number theory research we did during summer 2011. We attacked the problem of extending continued fractions to degrees higher than 2
	\end{itemize}

	Krishna Dasaratha et al. ``Cubic irrationals and periodicity via a family of multi-dimensional continued fraction algorithms." {\it Monatshefte f\"{u}r Mathematik} 174 (2014): 549-566. http://arxiv.org/abs/1208.4244
	\begin{itemize}  \itemsep -2pt
	\item Based on research done during summer 2011
	\end{itemize}


\section{EDUCATION}
	{\bf Coding House Institute}, Silicon Valley \hfill {\sl 2016}
	\begin{itemize}  \itemsep -2pt
	\item The ``Only Live-In" Web Dev Bootcamp
	\item Students eat, breathe, and sleep code for two intense months.  I stayed on for another two months as a Code Mentor.
	\end{itemize}

	{\bf Williams College}, Williamstown, MA \hfill {\sl B.A., 2013}
	\begin{itemize}  \itemsep -2pt
	\item Major: Mathematics \hfill GPA: 3.58
	\item Completed half the requirements for a Computer Science Major
	\end{itemize}


\hl\section{VOLUNTEER}
	{\bf Centro de Textiles Tradicionales del Cusco}, Peru \hfill {\sl 2015}
	\begin{itemize}  \itemsep -2pt
	\item English tutor \& Technology handyman
	\end{itemize}


\hl\section{LANGUAGES}
	{\bf English}, {\sl native} \\
	{\bf Spanish}, {\sl intermediate level -- lived in Peru 2014-2015}


\hl\section{MISC.}
	{\bf Appalachian Trail Thru-Hike} \hfill {\sl 2014}
	\begin{itemize}  \itemsep -2pt
	\item A 2200-mi.\ (3500-km.)\ footpath through the Appalachian Mountains
	\end{itemize}
	
	{\bf Wilderness First Aid}, NOLS Wilderness Medicine Institute \hfill {\sl 2014}
	\begin{itemize}  \itemsep -2pt
	\item Certification Course
	\end{itemize}

	{\bf Hudson River Undergraduate Math Conference}
	\begin{itemize}  \itemsep -2pt
	\item Presented on short topics during the 2009, 2010, 2011, and 2013 conferences
	\end{itemize}

	{\bf Joint Mathematics Meetings}, San Francisco, CA \hfill {\sl 2010}
	\begin{itemize}  \itemsep -2pt
	\item Presented the poster: {\sl The Isoperimetric Problem in Sectors with Density $r$}
	\item Wrote for the AMS Grad School Blog (http://blogs.ams.org/mathgradblog)
	\end{itemize}	
 
\end{resume}
\end{document}
