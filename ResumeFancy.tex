% LaTeX resume using res.cls
\documentclass[margin]{res}
%\usepackage[a4paper,hmargin=1.0in,vmargin=1.0in]{geometry}
\usepackage{url}
\setlength{\textwidth}{5.1in} % set width of text portion
%\setlength{\resumewidth}{6.5in} % set width of whole resume


% Draw a think horizontal line the whole width of resume:
\newcommand{\HL}{\moveleft\hoffset\vbox{\hrule width\resumewidth height 1pt}}
% Draw a thin horizontal line between sections:
\newcommand{\hl}{\moveleft\hoffset\vbox{\hrule width\resumewidth}\vspace{-0.11in}}


\begin{document}

% Center the name over the entire width of resume:
 \moveleft.5\hoffset\centerline{\huge\bf Nicholas G. Neumann-Chun}
 \vspace{0.08in}
 \moveleft.5\hoffset\centerline{\large\sl Full-Stack JavaScript Developer with a math degree from Williams College}
% \smallskip

\vspace{-0.02in} 
\HL % Horizontal line
\vspace{-0.12in}

% addresses begin here
\address{ 349 Harvard Common \\ Fremont, CA  94539 \\ (651) 491-4928 }
\address{nicholas.babelthaup@gmail.com \\ @Babelthuap \\ https://babelthuap.github.io }

% begin the resume content
\begin{resume}

\vspace{-0.1in}
\bigskip
\HL
\vspace{-0.13in}

\hl\section{COMPUTER}
	{\sl Skills:} Node, AngularJS, Express, MongoDB, JavaScript, jQuery, Gulp, Git, \LaTeX
	
	\vspace{-0.1in}
	
	{\sl Exposure:} Java, Python, Scala, Mathematica, ReactJS, Flux, GraphQL, Relay, jspm, Webpack, Mocha, Passport, Firebase, Heroku, Bootstrap, Foundation 
	

\section{EDUCATION}
	{\bf Coding House Institute}, San Francisco, CA \hfill {\sl expecting to finish February 2016}
	\begin{itemize}  \itemsep -2pt
	\item The ``Only Live-In" Web Dev Bootcamp
	\end{itemize}

	{\bf Williams College}, Williamstown, MA \hfill {\sl B.A., June 2013}
	\begin{itemize}  \itemsep -2pt
	\item Major: Mathematics \hfill GPA: 3.58
	\item Completed half the requirements for a Computer Science Major
	\end{itemize}

	{\bf Mounds Park Academy}, Maplewood, MN \hfill {\sl High School Diploma, June 2008}
%	\begin{itemize}  \itemsep -2pt
%	\item Highest marks (5) on Advanced Placement Spanish Exam
%	\item National Merit Scholar
%	\end{itemize}

	{\bf UMTYMP}, University of Minnesota Talented Youth Mathematics Program \hfill {\sl 2007}
	\begin{itemize}  \itemsep -2pt
	\item 7th -- 11th grade: Algebra -- Calculus III
	\end{itemize}


\hl\section{TEACHING \\ EXPERIENCE}
	{\bf Math TA}, Williams College \hfill {\sl 2011-2012}
	\begin{itemize}  \itemsep -2pt
	\item Fall 2011: Complex Analysis with Professor Frank Morgan
	\item Spring 2012: Real Analysis with Professor Mihai Stoiciu
	\end{itemize}
	
	{\bf Math/Physics Tutor}
	\begin{itemize}  \itemsep -2pt
	\item At Williams College, during the Fall 2009, Spring 2010, and Fall 2010 semesters
	\item While studying abroad in Burundi during January 2012
	\end{itemize}
	
	
\hl\section{LANGUAGES}
	{\bf English}, {\sl native} \\
	{\bf Spanish}, {\sl intermediate level -- lived in Peru for 6 months 2014-2015}


\hl\section{VOLUNTEER}
	{\bf Centro de Textiles Tradicionales del Cusco}, Peru \hfill {\sl January-May 2015}
	\begin{itemize}  \itemsep -2pt
	\item English tutor
	\item Technology handyman
	\end{itemize}


\hl\section{PUBLICATIONS}
	Garrity, Thomas. {\it Electricity and Magnetism for Mathematicians: A Guided Path from Maxwell's Equations to Yang-Mills}. New York: Cambridge University Press, 2015.
	\begin{itemize}  \itemsep -2pt
	\item Created all diagrams, including cover illustration, with Adobe Illustrator
	\end{itemize}
	
	Krishna Dasaratha, Laure Flapan, Thomas Garrity, Chansoo Lee, Cornelia Mihaila, Nicholas Neumann-Chun, Sarah Peluse, Matthew Stoffregen. ``A Generalized Family of Multidimensional Continued Fractions: TRIP Maps." {\it International Journal of Number Theory} 10.8 (2014): 2151-2186. \url{http://arxiv.org/abs/1206.7077}
	\begin{itemize}  \itemsep -2pt
	\item Based on research done during summer 2011
	\end{itemize}

	Krishna Dasaratha et al. ``Cubic irrationals and periodicity via a family of multi-dimensional continued fraction algorithms." {\it Monatshefte f\"{u}r Mathematik} 174 (2014): 549-566. \url{http://arxiv.org/abs/1208.4244}
	\begin{itemize}  \itemsep -2pt
	\item Based on research done during summer 2011
	\end{itemize}


\hl\section{OUTDOOR \\ EXPERIENCE}
	{\bf Superior Hiking Trail} \hfill {\sl 2015}
	\begin{itemize}  \itemsep -2pt
	\item 230 mi.\ along Lake Superior from the Canadian border to Two Harbors, MN
	\end{itemize}

	{\bf Appalachian Trail Thru-Hike} \hfill {\sl 2014}
	\begin{itemize}  \itemsep -2pt
	\item A 2200-mi.\ (3500-km.)\ footpath through the Appalachian Mountains
	\end{itemize}

	{\bf Wilderness First Aid}, NOLS Wilderness Medicine Institute \hfill {\sl October 2014}
	\begin{itemize}  \itemsep -2pt
	\item Certification Course
	\end{itemize}

	{\bf NOLS (National Outdoor Leadership School)}, Alaska \hfill {\sl July 2006}
	\begin{itemize}  \itemsep -2pt
	\item Month course
	\end{itemize}

	
\hl\section{MATH \\ CONFERENCE \\ ATTENDANCE}
	{\bf Hudson River Undergraduate Math Conference}
	\begin{itemize}  \itemsep -2pt
	\item Presented on short topics during the 2009, 2010, 2011, and 2013 conferences
%	\item 2009 Presentation: {\it Cryptology With Elliptic Curves}
%	\item 2010 Presentation: {\it Explicit Formulas for Operator Norms on Finite-Dimensional Vector Spaces}
%	\item 2011 Presentation: {\it Burnside's Theorem with an Application to Chemistry}
%	\item 2013 Presentation: {\it Life: Discrete or Continuous?}
	\end{itemize}

	{\bf International Soap Bubble Conference}, Edinburgh, Scotland \hfill {\sl March 2012}
	\begin{itemize}  \itemsep -2pt
	\item International Centre for Mathematical Sciences
	\end{itemize}

	{\bf Joint Mathematics Meetings}, San Francisco, CA \hfill {\sl January 2010}
	\begin{itemize}  \itemsep -2pt
	\item Presented the poster: {\sl The Isoperimetric Problem in Sectors with Density $r$}
	\item Wrote posts for the AMS Graduate School Blog (\url{http://blogs.ams.org/mathgradblog/})
	\end{itemize}	
 
\end{resume}
\end{document}
